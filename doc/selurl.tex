\paragraph{NAME\label{NAME}\index{NAME}}


selurl - Normalise and validate URIs for harvesting

\paragraph{INTRODUCTION\label{INTRODUCTION}\index{INTRODUCTION}}


Selurl selects and normalises URIs on basis of both general practice
(hostname lowercasing, portnumber substsitution etc.) and
Combine-specific handling (aplpying config\_allow, config\_exclude,
config\_serveralias and other relevant config settings).



The Config settings catered for currently are:



maxUrlLength - the maximum length of an unnormalised URL
allow - Perl regular to identify allowed URLs
exclude - Perl regular expressions to exclude URLs from harvesting
serveralias - Aliases of server names
sessionids - List sessionid markers to be removed



A selurl object can hold a single URL and has methods to obtain its
subparts as defined in URI.pm, plus some methods to normalise and
validate it in Combine context.

\paragraph{BUGS\label{BUGS}\index{BUGS}}


Currently, the only schemes supported are http, https and ftp. Others
may or may not work correctly. For one thing, we assume the scheme has
an internet hostname/port.



clone() will only return a copy of the real URI object, not a new
selurl.



URI URI-escapes the strings fed into it by new() once. Existing
percent signs in the input are left untouched, which implicates that:



(a) there is no risk of double-encoding; and



(b) if the original contained an inadvertent sequence that could
be interpreted as an escape sequence, uri\_unescape will not
render the original input (e.g. url\_with\_\%66\_in\_it goes whoop)
If you know that the original has not yet been escaped and wish to
safeguard potential percent signs, you'll have to escape them (and
only them) once before you offer it to new().



A problem with URI is, that its object is not a hash we can
piggyback our data on, so I had to resort to AUTOLOAD to emulate
inheritance. I find this ugly, but well, this *is* Perl, so what'd
you expect?

