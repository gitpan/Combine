\section{Evaluation of automated subject classification}
\label{autoclasseval}

Automated classification approach used for evaluation was \hyperref{string-matching of terms}{string-matching of terms (section }{)}{autoclass} from an engineering-specific controlled vocabulary Engineering Index, used in Elsevier's Compendex database. One reason for choosing this approach is that it does not require training documents that are often not available especially on the World Wide Web. For a discussion on different approaches to automated subject classification, see \cite{Golub_Review}. 

In \cite{ardo06:_ECDL} a machine-learning and a string-matching approach to automated subject classification of text were compared as to their performance on a test collection of six classes. It was shown that SVM on average outperforms the string-matching approach: our hypothesis that SVM yields better recall and string-matching better precision was confirmed only on one of the classes. The two approaches being complementary, we investigated different combinations of the two based on combining their vocabularies. SVM performs best using the original set of terms, and string-matching approach also has best precision when using the original set of terms. Best recall for string-matching is achieved when using descriptive terms. Reasons for these results need further investigation, including a larger data collection and combining the two using predictions. 

Another reason for choosing the string-matching approach is that Engineering Index is a well-developed vocabulary with an average of 88 manually selected terms designating one class \cite{GolubRole}. In this study, it has been explored to what degree different types of terms in Engineering Index influence automated subject classification performance. Preferred terms, their synonyms, broader, narrower, related terms, and captions were examined in combination with a stemmer and a stop-word list. The results showed that preferred terms perform best, whereas captions perform worst. Stemming in most cases showed to improve performance, whereas the stop-word list did not have a significant impact. The majority of classes is found when using all the terms and stemming: micro-averaged recall is 73%. The remaining 27% of classes were not found because the words in the term list designating the classes did not exist in the text of the documents to be classified. This study implies that all types of terms should be used for a term list in order to achieve best recall, but that higher weights could be given to preferred terms, captions and synonyms, as the latter yield highest precision. Stemming seems useful for achieving higher recall, and could be balanced by introducing weights for stemmed terms. Stop-word list could be applied to captions, narrower and preferred terms.

In \cite{ardo05:_ECDL} the was to determine how significance indicators assigned to different Web page elements (internal metadata, title, headings, and main text) influence automated classification. The data collection that was used comprised 1000 Web pages in engineering, to which Engineering Information classes had been manually assigned. The significance indicators were derived using several different methods: (total and partial) precision and recall, semantic distance and multiple regression. It was shown that for best results all the elements have to be included in the classification process. The exact way of combining the significance indicators turned out not to be overly important: using the F1 measure, the best combination of significance indicators yielded no more than 3% higher performance results than the baseline.

Other issues specific to Web pages were identified and discussed in \cite{GolubChall}. The focus of the study was a collection of Web pages in the field of engineering. Web pages present a special challenge: because of their heterogeneity, one principle (e.g. words from headings are more important than main text) is not applicable to all the Web pages of a collection. For example, utilizing information from headings on all Web pages might not give improved results, since headings are sometimes used simply instead of using bold or a bigger font size. A number of weaknesses of the described approach were identified, and ways to deal with those were proposed for further research. These include enriching the term list with synonyms and different word forms, adjusting the term weights and cut-off values and word-sense disambiguation.
