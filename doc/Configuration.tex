\section{Configuration}
\label{configuration}
Configuration files use a simple format consisting of either name/value pairs
or complex variables in sections. Name/value pairs are encoded as single lines 
formated like '{\tt name~=~value}'. Complex variables are encoded as multiple
lines in named sections delimited as in XML, using '{\tt <name> ... </name>}'.
Sections may be nested for related configuration variables.
Empty lines and lines starting with '{\tt \#}' (comments) are ignored.


The most important configuration variables are the complex variables
{\tt <url><allow>} (allows certain URLs to be harvested) and {\tt
<url><exclude>} (excludes certain URLs from harvesting) which are used
to limit your crawl to just a section of the WWW, based on the URL.
Loading URLs to be crawled into the system checks each URL first
against the Perl regular expressions of {\tt <url><allow>} and if it
matches goes on to match it against {\tt <url><exclude>} where it's
discarded if it matches, otherwise it's scheduled for crawling.
(See \hyperref{'URL filtering'}{section }{ 'URL filtering'}{urlfilt}).

\subsection{Configuration files}
All configuration files are stored in the {\tt /etc/combine/}
directory tree. All configuration variables have reasonable defaults (section \ref{configvars}).

\subsubsection{Templates}
The values in
\begin{description}
\item[job\_default.cfg] contains job specific defaults. It is copied to a
subdirectory named after the job by {\tt combineINIT}.

\item[SQLstruct.sql]
contains structure of the internal SQL database used both for administration
and for holding data records. \hyperref{Details}{Details in section }{}{sqlstruct}.

\item[Topic\_*] contains various contributed topic definitions.
\end{description}

\subsubsection{Global configuration files}
Files used for global parameters for all crawler jobs.
\begin{description}

\item[default.cfg] is the global defaults. It is loaded first.
Consult \hyperref{'Configuration Variables'}{section }{}{configvars}
and \hyperref{'Default configuration files'}{appendix }{}{conffiles} for details.
Values can be overridden from
the job-specific configuration file {\tt combine.cfg}.

\item[tidy.cfg] configuration for Tidy cleaning of HTML code.

\end{description}

\subsubsection{Job specific configuration files}
The program {\tt combineINIT} creates
a job specific sub-directory in {\tt /etc/combine} and populates it with some files including {\tt combine.cfg}
initialized with a copy of {\tt job\_default.cfg}.
You should always change the value of the variable {\tt Operator-Email} 
in this file and set it to something
reasonable. It is used by Combine to identify you to the crawled Web-servers.

The job-name have to be given to all programs
when started using the \verb+--jobname+ switch. 

\begin{description}

\item[combine.cfg] the job specific configuration. It is loaded second
 and overrides the global defaults. Consult \hyperref{section 'Configuration Variables'}{section }{}{configvars}
and \hyperref{'Default configuration files'}{appendix }{}{conffiles} for details.

\item[topicdefinition.txt] contains the topic definition for
focused crawl if the \verb+--topic+ switch is given to {\tt combineINIT}.
The format of this file is described in \hyperref{'Topic definition'}{section }{}{topicdef}.

\item[stopwords.txt] a file with words to be excluded from the automatic topic
classification processing. One word per line. Can be empty (default) but must be present.

\item[config\_exclude] contains more exclude patterns.
Optional, automatically included by {\tt combine.cfg}. Updated by {\tt combineUtil}.

\item[config\_serveralias] contains patterns for resolving Web server aliases.
Optional, automatically included by {\tt combine.cfg}. Updated by {\tt combineUtil}.
\item[sitesOK.txt] optionally used by the
 \hyperref{built-in automated classification algorithms}{built-in automated classification algorithms (section }{)}{autoclass} to bypass
the topic filter for certain sites.
\end{description}

\subsubsection{Details and default values}
Further details are found in
\hyperref{'Configuration variables'}{section }{ 'Configuration variables'}{configvars} which lists
all variables and their default values.
