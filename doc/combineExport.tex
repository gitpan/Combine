\paragraph{NAME\label{NAME}\index{NAME}}


combineExport - export records in XML from Combine database

\paragraph{SYNOPSIS\label{SYNOPSIS}\index{SYNOPSIS}}


combineExport --jobname $<$name$>$ [--profile alvis$|$dc$|$combine --charset utf8$|$isolatin --number $<$n$>$ --recordid $<$n$>$ --md5 $<$MD5$>$ --incremental --xsltscript ...]

\paragraph{OPTIONS AND ARGUMENTS\label{OPTIONS_AND_ARGUMENTS}\index{OPTIONS AND ARGUMENTS}}


jobname is used to find the appropriate configuration (mandatory)

\begin{description}

\item[{--profile}] \mbox{}

Three profiles: alvis, dc, and combine . alvis and combine are similar XML formats.



'alvis' profile format is defined by the Alvis enriched document format DTD.
It uses charset UTF-8 per default.



'combine' is more compact with less redundancy.



'dc' is XML encoded Dublin Core data.


\item[{--charset}] \mbox{}

Selects a specific characterset from UTF-8, iso-latin-1
Overrides --profile settings.


\item[{--collapseinlinks}] \mbox{}

Skip inlinks with duplicate anchor-texts (ie just one inlink per unique anchor-text).


\item[{--nooutlinks}] \mbox{}

Do not include any outlinks in the exported records.


\item[{--ZebraIndex}] \mbox{}

ZebraIndex sends XML records directly to the Zebra server defined in Combine
configuration variable 'ZebraHost'.
It uses the default Zebra configuration: profile=combine, nooutlinks, collapseinlinks
and is compatible with the direct Zebra indexing done during harvesting when
'ZebraHost' is defined in the Combine configuration. Requires that the Zebra server
is running.


\item[{--SolrIndex}] \mbox{}

SolrIndex sends XML records directly to the Solr server defined in Combine
configuration variable 'SolrHost'.
It uses the default Solr configuration: profile=combine, nooutlinks, collapseinlinks
and is compatible with the direct Solr indexing done during harvesting when
'SolrHost' is defined in the Combine configuration. Requires that the Solr server
is running.


\item[{--xsltscript}] \mbox{}

Generates records in Combine native format and converts them using this XSLT script
before output. See example scripts in /etc/combine/*.xsl


\item[{--number}] \mbox{}

the max number of records to be exported


\item[{--recordid}] \mbox{}

Export just the one record with this recordid


\item[{--md5}] \mbox{}

Export just the one record with this MD5 checksum


\item[{--pipehost, --pipeport}] \mbox{}

Specifies the server-name and port to connect to and export data using the Alvis Pipeline.
Exports incrementally, ie all changes since last call to combineExport with the same pipehost
and pipeport.


\item[{--incremental}] \mbox{}

Exports incrementally, ie all changes since last call to combineExport using --incremental

\end{description}
\paragraph{DESCRIPTION\label{DESCRIPTION}\index{DESCRIPTION}}
\paragraph{EXAMPLES\label{EXAMPLES}\index{EXAMPLES}}
\begin{verbatim}
 Export all records in Alvis XML-format to the file recs.xml
   combineExport --jobname atest > recs.xml
\end{verbatim}
\begin{verbatim}
 Export 10 records to STDOUT
   combineExport --jobname atest --number 10
\end{verbatim}
\begin{verbatim}
 Export all records in UTF-8 using Combine native format
   combineExport --jobname atest --profile combine --charset utf8 > Zebrarecs.xml
\end{verbatim}
\begin{verbatim}
 Incremental export of all changes from last call using localhost at port 6234 using the
 default profile (Alvis)
   combineExport --jobname atest --pipehost localhost --pipeport 6234
\end{verbatim}
\paragraph{SEE ALSO\label{SEE_ALSO}\index{SEE ALSO}}


Combine configuration documentation in \emph{/usr/share/doc/combine/}.



Alvis XML schema (--profile alvis) at
\textsf{http://project.alvis.info/alvis\_docs/enriched-document.xsd}

\paragraph{AUTHOR\label{AUTHOR}\index{AUTHOR}}


Anders Ard�, $<$anders.ardo@it.lth.se$>$

\paragraph{COPYRIGHT AND LICENSE\label{COPYRIGHT_AND_LICENSE}\index{COPYRIGHT AND LICENSE}}


Copyright (C) 2005 - 2006 Anders Ard�



This library is free software; you can redistribute it and/or modify
it under the same terms as Perl itself, either Perl version 5.8.4 or,
at your option, any later version of Perl 5 you may have available.

\begin{verbatim}
 See the file LICENCE included in the distribution at
 L<http://combine.it.lth.se/>
\end{verbatim}
