\section{System components}
\label{comp}
All executables take a mandatory switch {\tt --jobname} which is used
to identify the particular crawl job you want and as well the
job-specific configuration directory.

Briefly {\tt combineINIT} is used to initialize SQL database and the
job specific configuration directory. {\tt combineCtrl} controls a Combine crawling job (start, stop, etc) as well as printing some statistics. {\tt combineExport} export records in various XML formats and 
{\tt combineUtil} provides various utility operations on the Combine database.

\hyperref{Detailed dependency information}{Detailed dependency 
information (section }{)}{moddep} can be found in the 'Gory
details' section.

\subsection{combineINIT}
Creates a Mysql database, database tables and initializes it. If the database exists it is dropped and recreated. A job specific configuration directory is created in {\tt /etc/combine/} and populated with a default configuration file.

If a topic definition filename is given, focused crawling using this topic defintion is enabled per default. Otherwise focused crawling is disabled, and Combine works as a general crawler.

\subsection{combineCtrl}

Implements various control functionality to administer a crawling job, like starting and stoping crawlers, injecting URLs into the crawl que, scheduling newly found links for crawling, controlling scheduling, etc.

This is the preferred way of controling a crawl job.

\subsection{combineUtil}
Does various statistics generation as well as performing sanity checks on the database.

\subsection{combineExport}
Export is done according to one of three profiles: {\tt alvis},
{\tt dc}, or {\tt combine}. {\tt alvis} and {\tt combine} are
very similar XML formats where {\tt combine} is more compact with
less redundancy and
{\tt alvis} contains some more information.
{\tt dc} is XML encoded Dublin Core data.

The {\tt alvis} profile format is defined by the  \htmladdnormallinkfoot{Alvis Enriched Document XML Schema}{http://www.miketaylor.org.uk/tmp/alvis/d3.1/enriched-document.xsd}.

For convinience a switch {\tt--xsltscript} adds the possibility
to filter the output using a XSLT script. The script is feed
a record according to the {\tt combine} profile and the result
is exported.
 
\subsection{Internal executables and Library modules}
{\tt combine} is the main crawling machine in the Combine system and 
{\tt combineRun} starts, monitors and restarts {\tt combine} crawling processes.

%\subsubsection{combine}
%\subsubsection{combineRun}
\subsubsection{Library}

Main, crawler specific, library components are collected in the {\tt Combine::} Perl namespace.